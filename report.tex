
\documentclass[a4paper,12pt,twoside]{report}
\usepackage[left=2cm,right=2cm,top=2cm,bottom=3cm]{geometry}


\documentclass[a4paper,12pt,twoside]{report}
\usepackage[left=2cm,right=2cm,top=2cm,bottom=3cm]{geometry}


\documentclass[a4paper,12pt,twoside]{report}
\usepackage[left=2cm,right=2cm,top=2cm,bottom=3cm]{geometry}


\documentclass[a4paper,12pt,twoside]{report}
\usepackage[left=2cm,right=2cm,top=2cm,bottom=3cm]{geometry}

\include{report.preamble}


\begin{document}

\title{\LARGE {\bf Exploiting the Human Computational Effort Dedicated to Message Reply Formatting for Training Discursive Email Segmenters}\\
 \vspace*{6mm}
}

\author{Soufian Salim}
\submitdate{July 2014}

\normallinespacing
\maketitle

\preface

\addcontentsline{toc}{chapter}{Abstract}

\begin{abstract}

In the context of multi-domain and multimodal online asynchronous discussion analysis, we propose an innovative strategy for manual annotation of dialog act (DA) segments. The process aims at supporting the analysis of messages in terms of DA. Our objective is to train a sequence labelling system to detect the segment boundaries.  The originality of the proposed approach is to avoid manually annotating the training data and instead exploit the human computational efforts dedicated to message reply formatting  when the writer replies to a message by inserting his response just after the quoted text appropriate to his intervention. We describe the approach, propose a new electronic mail corpus and report the evaluation of segmentation models we built.

\end{abstract}

\cleardoublepage

\addcontentsline{toc}{chapter}{Acknowledgements}

\begin{acknowledgements}

I would like to express gratitude to:

\begin{itemize}
 \item Nicolas Hernandez
 \vspace*{3mm}
 \item Christian Viard-Gaudin
 \vspace*{3mm}
 \item TALN team members at LINA
 \vspace*{3mm}
 \item ATAL teachers and students
\end{itemize}

\end{acknowledgements}

\body

\input{sections/introduction}

\input{sections/background_and_related_work}

\input{sections/methodology_for_automatic_corpora_annotation}

\input{sections/methodology_for_email_segmentation}

\input{sections/experimental_framework}

\input{sections/experiments_and_results}

\input{sections/conclusion}

\appendix
% appendices come here



\addcontentsline{toc}{chapter}{Bibliography}
\bibliographystyle{alpha}
\bibliography{bibliography}

\end{document}



\begin{document}

\title{\LARGE {\bf Exploiting the Human Computational Effort Dedicated to Message Reply Formatting for Training Discursive Email Segmenters}\\
 \vspace*{6mm}
}

\author{Soufian Salim}
\submitdate{July 2014}

\normallinespacing
\maketitle

\preface

\addcontentsline{toc}{chapter}{Abstract}

\begin{abstract}

In the context of multi-domain and multimodal online asynchronous discussion analysis, we propose an innovative strategy for manual annotation of dialog act (DA) segments. The process aims at supporting the analysis of messages in terms of DA. Our objective is to train a sequence labelling system to detect the segment boundaries.  The originality of the proposed approach is to avoid manually annotating the training data and instead exploit the human computational efforts dedicated to message reply formatting  when the writer replies to a message by inserting his response just after the quoted text appropriate to his intervention. We describe the approach, propose a new electronic mail corpus and report the evaluation of segmentation models we built.

\end{abstract}

\cleardoublepage

\addcontentsline{toc}{chapter}{Acknowledgements}

\begin{acknowledgements}

I would like to express gratitude to:

\begin{itemize}
 \item Nicolas Hernandez
 \vspace*{3mm}
 \item Christian Viard-Gaudin
 \vspace*{3mm}
 \item TALN team members at LINA
 \vspace*{3mm}
 \item ATAL teachers and students
\end{itemize}

\end{acknowledgements}

\body


\chapter{Introduction}

\label{ch:introduction}

\section{Contexte}

\lipsum[1]

\section{Motivation et objectifs}

\lipsum[1]

\section{Contributions}

\lipsum[1]


\chapter{Concepts et étude bibliographique}

\label{ch:background_and_related_work}

\section{Actes du langage}

\lipsum[1]

\section{Segmentation de textes}

\lipsum[1]

\section{Corpus de courriers électroniques}

\lipsum[1]


\chapter{Étiquetage automatique de corpus}

\label{ch:methodology_for_automatic_corpora_annotation}

Dans ce chapitre, nous présentons notre hypothèse ainsi que les étapes détaillées de notre approche pour l'étiquetage automatique de corpus.

\section{Hypothèses}

Nous partons du postulat que, lorsqu'un internaute reprend certains passages du courriel auquel il répond dans son message, il effectue des opérations cognitives pour identifier des fragments de texte autonomes et homogènes. Ces opérations peuvent être interpretées comme des opérations d'annotation. Les suppositions que l'on peut faire sur le type d'annotation dont il s'agit dépendent de l'opération qui a été effectuée. La suppression ou la reprise de texte original peut donner des indices sur la pertinence du contenu : du texte rejeté est probablement moins pertinent que du texte réutilisé.

\section{Schéma d'annotations}

Nous supposons que lorsqu'une personne ajoute du nouveau contenu entre deux blocs de texte cité, il effectue un découpage du message original. Nous supposons que la partie citée consiste en une unité d'information homogène. Par exemple, on peut supposer que la première phrase d'une partie citée comporte des instructions pour ouvrir un nouveau segment de discours tandis que la dernière phrase comporte des instructions pour achever le seegment. Par conséquent, nous pouvons effectuer certaines suppositions par rapport au role joué par ces phrases dans la structure informationnelle du message d'origine. On suppose qu'un message peut être divisé en segments du discours subsequents et consécutifs, chacun porteur de son propre acte du dialogue. On prend la phrase comme unité élémentaire. Une phrase dans un segment peut jouer l'un des roles suivants: \emph{starting and ending} (\textit{SE}), si elle constitue un segment à elle seule, \emph{starting} (\textit{S}), si elle débute un segment, \emph{inside} (\textit{I}), si elle n'est ni en début ni en fin de segment, et \emph{ending} (\textit{E}), si elle termine un segment.

Ce schéma est similaire au schéma \emph{BIO} à la différence qu'il est appliqué au niveau de la phrase et non au niveau du token \cite{ratinov:2009:conll}.

La figure~\ref{fig:exampleSegmentationLabels} illustre ce schéma en montrant comment les phrases de la figure~\ref{fig:exampleSourceReplyMessage} peuvent être alignées et comment les étiquettes peuvent en être inférées.

\section{Procédure de génération des données annotées}

Avant de pouvoir prédire les labels des phrases du message originel, il est nécessaire d'identifier celles qui ont été réutilisées dans un message de réponse. L'identification des lignes citées dans le message de réponse est insuffisant pour diverses raisons.

Premièrement, le segmenteur est supposé fonctionner sur des données non-bruitées (i.e. les nouveaux contenus dans les messages) alors qu'un texte cité est une version altérée du texte original. En effet, certains clients de messagerie électronique ne respectent pas toujours les standards et ne sont pas focément toujours compatibles\footnote{Les \textit{Request for Comments} (RFC) sont des règles et protocoles proposés par les groupes de travail participant à l'\textit{Internet Standardization} (\url{https://tools.ietf.org/html}). Certains RFC sont consacrés aux formats des courriels et aux spécifications d'encodage (voir RFC 2822 et 5335 pour commencer). Il y a eu de nombreuses propositions, parfois mises à jours et donc parfois rendues caduques, ce qui peut expliquer certains problèmes de compatibilité)}. En particulier, l'absence de certaines métadonnées peut causer le mauvais ré-encodage des blocs de citation à chaque échange. De plus, les programmes clients peuvent intégrer leurs propres mécanismes pour citer les précédents messages, ou encore tronquer les lignes trop longues\footnote{Fonctionnalité utilisée pour rendre le texte lisible sans avoir à scroller horizontalement. Les phrases sont généralement découpées en segments d'environ 80 caractères.}.

Deuxièmement, accéder aux messages originels peut permettre de prendre en compte certains traits contextuels (comme la disposition visuelle par exemple).

Troisièmement, pour aller plus loin, le context original du texte extrait contient également de l'information sur la segmentation d'un message. Par exemple, une phrase du message originel, qui ne serait pas présente dans la réponse, mais qui suit une phrase alignée, peut être considérée comme débutant un nouveau segment.

Don, en plus d'identifier les lignes citées, nous déployons une procédure d'alignement pour obtenir la version originale du texte cité. La procédure décrite étiquette les phrases de messages sources avec une information sur leur segmentation. Elle suit les étapes suivantes :

\begin{enumerate}
    \item Les messages postés dans le style interfolié sont identifiés
    \item Pour chaque paire message source / réponse :
    \begin{enumerate}
        \item Les deux messages sont tokenisés au niveau de la phrase et du mot (voir sous-section~\ref{subsec:tokenization} pour le détail des techniques employées pour la tokenisation)
        \item Les lignes de citations dans la réponse sont identifiées
        \item Les phrases qui font partie du texte cité dans le message de réponse sont identifiées
        \item Les phrases du message d'origine sont alignées avec le texte cité dans la réponse (voir sous-section~\ref{subsec:tokenization} pour le détail de la procédure d'alignement)
        \item Les phrases alignées sont étiquetées (voir sous-section~\ref{subsec:labelling} pour le détail de l'algorithme d'étiquetage)
        \item La séquence de phrases alignées est ajoutée au jeu de données
    \end{enumerate}
\end{enumerate}

Les messages contenant des messages à contenu interfolié sont reconnus grâce à la présence d'au moins deux lignes citées consécutives séparées par des lignes de nouveau contenu. Les paires de messages sources et leur réponse sont constituées à partir des champs \emph{in-reply-to} de leurs entêtes. Comme déclaré dans le RFC 3676\footnote{\url{http://www.ietf.org/rfc/rfc3676.txt}}, nous considérons comme des lignes citées les lignes commençant par le symbole "\textgreater" (chevron). Les lignes qui ne sont pas des lignes citées sont considérées comme étant des nouvelles lignes. Les tokens sont utilisés pour indexer les lignes citées et les phrases.

\subsection{Tokenization}

\label{subsec:tokenization}

\subsection{Alignement}

\label{subsec:alignment}

Pour trouver les alignements entre deux messages donnés, nous utilisons un algorithme d'alignement de chaînes basé sur la programmation dynamique (DP) \cite{sankoff:1983}. Dans le contexte de la reconnaissance de la parole, cet algorithme est aussi connu sous le nom de \textit{NIST align/scoring algorithm}. En effet il est largement utilisé pour évaluer les systèmes de reconnaissance de la parole en comparant leurs sorties au texte de référence. Il est utilisé en particulier pour calculer deux taux d'erreur : le \textit{Word Error Rate} (WER) et le \textit{Sentence Error rate} (SER).

L'algorithme fonctionne en effectuant une réduction de la distance de Levenshtein en attribuant aux mots corrects, aux insertions, aux suppressions et aux substitutions des poids respectifs de 0, 75, 75 et 100. L'algorithme est de complexité $O(MN)$.

L'Université de Carnegie Mellon fournit une implémentation de cet algorithme dans son kit de reconnaissance de la parole\footnote{Sphinx 4, $edu.cmu.sphinx.util.NISTAlign$, \url{http://cmusphinx.sourceforge.net}}

\subsection{Étiquetage}

\label{subsec:labelling}

L'étiquetage d'une phrase alignée (phrase du message source réutilisée dans la réponse) se fait suivant un simple algorithme à base de règles :

\begin{itemize}
    \item[\bullet] Pour chaque phrase source alignée :
    \begin{itemize}
        \item[\bullet] si la phrase est entourée par du nouveau contenu dans la réponse, l'étiquette est \texttt{Start\&End}
        \item[\bullet] sinon si la phrase est précédée par du nouveau contenu, l'étiquette est \texttt{Start}
        \item[\bullet] sinon si la phrase est suivie par du nouveau contenu, l'étiquette est \texttt{End}
        \item[\bullet] sinon, l'étiquette est \texttt{Inside}
    \end{itemize}
\end{itemize}

\begin{figure}
    \begin{minipage}{\textwidth}
        \fbox {
            \parbox{\linewidth}{
                \vspace{3mm}
                \small
                [Hi!]$^{S1}$\vspace{0.3cm}

                [I got my ubuntu cds today and i'm really impressed.]$^{S2}$ [My friends like them and my teachers too (i'm a student).]$^{S3}$ [It's really funny to see, how people like ubuntu and start feeling geek and blaming microsoft when they use it.]$^{S4}$ \vspace{0.3cm}

                [Unfortunately everyone wants an ubuntu cd, so can i download the cd covers anywhere or an 'official document' which i can attach to self-burned cds?]$^{S5}$\vspace{0.3cm}

                [I searched the entire web site but found nothing.]$^{S6}$ [Thanks in advance.]$^{S7}$\vspace{0.3cm}

                [John]$^{S8}$

                \vspace{3mm}
            }
        }

        \begin{center}
        Message source.
        \end{center}
        
        \fbox {
            \parbox{\linewidth}{
                \vspace{3mm}
                \small
                [On Sun, 04 Dec 2005, John Doe 
                \textless john@doe.com\textgreater wrote:]$^{R1}$\vspace{0.3cm}

                \textgreater [I got my ubuntu cds today and i'm really impressed.]$^{R2}$ [My friends like them and \\ \
                \textgreater my teachers too (i'm a student).]$^{R3}$ [It's really funny to see, how people like ubuntu \\ \
                \textgreater and start feeling geek and blaming microsoft when they use it.]$^{R4}$\vspace{0.3cm}

                [Rock!]$^{R5}$\vspace{0.3cm}

                \textgreater [Unfortunately everyone wants an ubuntu cd, so can i download the cd covers \\ \ 
                \textgreater anywhere or an 'official document' which i can attach to self-burned cds?]$^{R6}$\vspace{0.3cm}

                [We don't have any for the warty release, but we will have them for hoary, %\\ \ 
                because quite a few people have asked. :-)]$^{R7}$\vspace{0.3cm}

                [Bob.]$^{R8}$ %\vspace{0.1cm}
                \vspace{3mm}
            }
        }
        
        \begin{center}
        Message de réponse.
        \end{center}
    \end{minipage}

    \caption{Un message originel (ou "message source") et sa réponse (tirés de l'archive de courriers électroniques \textit{ubuntu-users}). Les différentes phrases ont été clairement indiquées.}
    \label{fig:exampleSourceReplyMessage}
\end{figure}

\begin{figure}
    \begin{minipage}{\textwidth}
        \small\centering
        \begin{tabular}{*{2}{|l}|c|}
        \hline
        \textbf{Source} & \textbf{Réponse} & \textbf{Étiquette}\\
            \hline
            S1  & & \\
            & R1 & \\
            \textit{S2}  & \textgreater \textit{R2}& \texttt{Start}\\
            \textit{S3}  & \textgreater \textit{R3}& \texttt{Inside}\\
            \textit{S4}  & \textgreater \textit{R4}& \texttt{End}\\
            & R5 & \\
            \textit{S5}  & \textgreater \textit{R6} & \texttt{Start\&End}\\
            & R7 & \\
            & [...] & \\
            S6 &  & \\ \ 
            [...] &  & \\
            \hline
        \end{tabular}
    \end{minipage}

    \caption{Alignement des phrases tirées des messages montrés dans la figure~\ref{fig:exampleSourceReplyMessage}, ainsi que les étiquettes inferrées de la reprise de texte du message source. Les étiquettes sont associées au phrases d'origine.}
    \label{fig:exampleSegmentationLabels}
\end{figure}


\chapter{Segmentation de courriels}

\label{ch:methodology_for_email_segmentation}

Dans ce chapitre, nous présentons notre approche pour la segmentation de courriels ainsi que les traits utilisés pour l’entraînement du classifieur.

\section{Étiquetage de séquences}

Nous choisissons de traiter le problème de la segmentation comme une tache d'étiquetage de séquences dont l'objectif est d'attribuer globalement le meilleur ensemble d'étiquettes pour la séquence entière d'un seul coup\footnote{Un exemple classique de tâche accomplie de cette manière est l'étiquetage morpho-syntaxique, qui cherche à identifier la nature grammaticale de chaque terme d'une phrase ou d'un document.}. Dans cette perspective, chaque courriel est traité comme une séquence de phrases. L'idée sous-jacente est que l'étiquette la plus pertinente pour une phrase est dépendante des traits et de l'étiquette des phrases proches. 

Notre segmenteur est basé sur un classifieur utilisant les champs aléatoires de Markov, tel qu'implémenté dans le programme d'étiquetage de séquences \textit{Wapiti} \cite{lavergne2010practical}. Nous fixons la taille de la fenêtre à 5, c'est à dire que l'algorithme prend en compte non seulement les traits de la phrase qu'il cherche à étiqueter mais également ceux des deux phrases précédentes et des deux phrases suivantes.

Entraîner le classifieur à reconnaître les différents labels du schéma d'annotation précédemment déterminé peut être problématique. En effet, il présente certains inconvénients qui peut nuire à l'efficacité du classifieur. En particulier, les phrases étiquetées \textit{SE} partageront, par définition, d'importantes caractéristiques avec les phrases étiquetées \textit{S} et \textit{E}. Nous choisissons donc de transformer ces annotations en un schéma binaire et nous contentons de différencier les phrases qui débutent un nouveau segment (\textit{True}), ou "phrases-frontières", de celles qui ne débutent pas un nouveau segment (\textit{False}). Le processus de conversion est trivial, et peut facilement être inversé.

Procédure de conversion :

\begin{itemize}
    \item[] Pour chaque phrase :
    \begin{itemize}
        \item[(a)] si la phrase est étiquetée \textit{SE} ou \textit{S}, l'étiquette devient \textit{True}
        \item[(b)] sinon, elle devient \textit{False}
    \end{itemize}
\end{itemize}

Procédure inverse, pour retrouver les étiquettes d'origine :

\begin{itemize}
    \item[] Pour chaque phrase :
    \begin{itemize}
        \item[(a)] si l'étiquette de la phrase courante est \textit{True} :
	    \begin{itemize}
	        \item[(i)] si la phrase suivante est étiquetée \textit{True}, elle devient \textit{SE}
	        \item[(ii)] sinon, elle devient \textit{S}
	    \end{itemize}
        \item[(b)] sinon :
	    \begin{itemize}
	        \item[(i)] si la phrase suivante est étiquetée \textit{True}, elle devient \textit{E}
	        \item[(ii)] sinon, elle devient \textit{I}
	    \end{itemize}
    \end{itemize}
\end{itemize}

\section{Ensembles de traits}

On distingue cinq ensembles de traits : les $n$-grammes, les traits basés sur la théorie de la structure de l'information, les traits thématiques, les traits stylistiques et les traits sémantiques (dans le cadre des expériences, les deux derniers ensembles sont regroupés sous l’appellation "traits divers"). Tous les traits sont indépendants du domaine et presque tous les traits sont indépendants du langage, à l'exception des traits sémantiques, qui peuvent néanmoins être facilement traduits.

Pour construire les traits du segmenteur, nous utilisons l'étiqueteur de Stanford pour l'étiquetage morpho-syntaxique \cite{toutanova2003feature}, et la base de données lexicale \textit{WordNet} pour la lemmatisation \cite{miller1995wordnet}.

\subsection{$n$-grammes}

On sélectionne, de manière insensible à la casse, les 1000\footnote{Valeur estimée empiriquement.} bigrammes et trigrammes apparaissant dans le plus grand nombre de phrases du corpus (ou \textit{document frequency}). Puisque la probabilité d'avoir de multiples occurrences d'un même $n$-gramme dans une phrase est extrêmement faible, nous ne conservons pas le nombre d'occurrences mais une valeur booléenne pour ne considérer que la présence ou l'absence du $n$-gramme.

\subsection{Traits basés sur la théorie de la structure de l'information}

Cet ensemble de traits est inspiré de la théorie de la structure de l'information, qui décrit l'information portée par une phrase en fonction de la façon dont elle est reliée à son contexte \cite{kruijff:1996}. La théorie affirme l'importance de constructions syntaxiques particulières et de l'ordre des mots dans la phrase. En effet pour des langages comme l'anglais ou le français, le début de la phrase est une position importante pour structurer l'information au niveau du discours, tandis que la fin de la phrase peut comporter de l'information utile pour annoncer ce qui vient ensuite. 

On s'intéresse aux trois premiers et derniers tokens significatifs de la phrase. Un token est considéré comme significatif si sa fréquence est supérieure à 1/2 000\footnote{Cette valeur a été déterminée empiriquement par rapport à nos données. Un travail supplémentaire devra être effectué pour la généraliser.}. Si une phrase contient moins de six tokens significatifs, le même token peut se retrouver dans les deux triplets. Si la phrase contient moins de trois tokens significatifs, les valeurs manquantes sont remplacées par une valeur spéciale "bouche-trou". Nous définissons trois traits individuels pour chacun des trois unigrammes, les deux bigrammes et le trigramme qui se trouvent dans chacun de ces triplets. Les traits sont les suivants : la forme de surface de chaque token (sensible à la casse), leur forme lemmatisée (insensible à la casse) et leur étiquette morpho-syntaxique. Ces traits sont illustrés par la figure \ref{fig:exampleSyntacticFeatures}.

\subsection{Trait thématique}

Le seul trait que nous prenons en compte pour la reconnaissance des variations thématiques est la sortie de l'algorithme \textit{TextTiling} \cite{hearst1997texttiling}. \textit{TextTiling} est l'un des algorithmes les plus communément utilisés pour la segmentation automatique de texte. Si l'algorithme détecte une rupture dans la cohésion lexicale du texte (entre deux blocs consécutifs), il place une frontière pour indiquer un changement thématique. En raison de la taille relativement courte des courriels, nous définissons la taille d'un bloc comme égale à trois fois la taille moyenne d'une phrase dans notre corpus. Nous définissons la "taille-étape" (la distance parcouru par la fenêtre glissante à chaque étape) comme égale à la taille moyenne d'une phrase du corpus.

\subsection{Traits divers}

Cet ensemble inclut les traits stylistiques et sémantiques. Il contient 24 traits, plusieurs ayant été empruntés à des travaux dans le domaine de la classification d'actes de dialogue \cite{qadir2011classifying} et de la segmentation de courriels \cite{lampert2009segmenting}. 

Les traits stylistiques capturent l'information portant sur la structure visuelle et la composition du message : 

\begin{itemize}
	\item[$\bullet$] la position de la phrase dans le courriel
	\item[$\bullet$] la taille moyenne des tokens
	\item[$\bullet$] le nombre total de tokens 
	\item[$\bullet$] le nombre total de caractères
	\item[$\bullet$] la proportion de majuscules
	\item[$\bullet$] la proportion de caractères alphabétiques
	\item[$\bullet$] la proportion de caractères numériques
	\item[$\bullet$] le nombre de chevrons
	\item[$\bullet$] si la phrase finit sur ou contient un point d'interrogation, une virgule ou un point-virgule
	\item[$\bullet$] si la phrase contient des caractères de ponctuation parmi ses trois premiers tokens (pour reconnaître les salutations \cite{qadir2011classifying}).
\end{itemize}

Les traits sémantiques cherchent à identifier certains mots et formules particuliers : 

\begin{itemize}
	\item[$\bullet$] si la phrase commence par un mot interrogatif de type "wh" (\textit{``who''}, \textit{``when''}, \textit{``where''}, \textit{``what''}, \textit{``which''}, \textit{``what''}, \textit{``how''})
	\item[$\bullet$] si la phrase contient un mot interrogatif de type "wh"
	\item[$\bullet$] si la phrase commence par une forme interrogative (e.g. "\textit{is it}", "\textit{are there}"...)
	\item[$\bullet$] si la phrase contient une forme interrogative
	\item[$\bullet$] si la phrase contient un modal (\textit{``can''}, \textit{``may''}, \textit{``must''}, \textit{``shall''}, \textit{``will''}, \textit{``might''}, \textit{``should''}, \textit{``would''}, \textit{``could''}, et leurs formes négatives)
	\item[$\bullet$] si la phrase contient une formule de planification (e.g. "\textit{I will}", "\textit{we are going to}"...)
	\item[$\bullet$] si la phrase contient des indices de la première personne (e.g. "\textit{we}", "\textit{my}"...)
	\item[$\bullet$] si la phrase contient des indices de la deuxième personne
	\item[$\bullet$] si la phrase contient des indices de la troisième personne
	\item[$\bullet$] le premier pronom personnel trouvé dans la phrase
	\item[$\bullet$] la première forme verbale rencontrée, telle qu'étiquetée par l'étiqueteur de Stanford, c'est à dire un élément du \textit{Penn Treebank tag set}\footnote{Liste alphabétique des étiquettes morpho-syntaxiques utilisées par le \textit{Penn Treebank Project} : \url{http://www.ling.upenn.edu/courses/Fall_2003/ling001/penn_treebank_pos.html}} (e.g. le trait \textit{``VBZ''} indique un verbe au présent et à la troisième personne du singulier).
\end{itemize}

\begin{table}\small\centering
	\begin{tabular}{*{2}{c}c}
		\toprule
		\textbf{Formes de surface} & \textbf{Lemmes} & \textbf{Étiquettes}\\
		\midrule
		Many & many & JJ\\
		thanks & thanks & NNS\\
		to & to & TO\\
		your & your & PRP\\
		suggestions & suggestion & DD\\
		. & . & .\\
		Many thanks & many thanks & JJ NNS\\
		thanks to . & thanks to . & NNS TO . \\
		your suggestions & your suggestion & PRP DD\\
		suggestions & suggestion & DD\\
		Many thanks to & many thanks to & JJ NNS TO\\
		your suggestions . & your suggestion . & PRP DD .\\
		\bottomrule
	\end{tabular}
	\caption{Traits syntaxiques formés par la phrase "\textit{Many thanks to all of you for the help you have offered, I have learned tremendously from all your suggestions}". Chaque cellule est un trait (36 au total).}
	\label{fig:exampleSyntacticFeatures}
\end{table}


\chapter{Experimental Framework}

\section{Corpora}

\subsection{Ubuntu Corpus}

Ubuntu Corpus.

\subsection{BC3 Corpus}

BC3 Corpus.

\section{Evaluation Protocol}

Evaluation Protocol.

\subsection{Methodology}

Methodology.

\subsection{Baselines}

Baselines.

\subsection{Metrics}

Metrics.


\chapter{Experiments and Results}

\section{10-fold cross validation}

\subsection{Preprocessing}

Preprocessing.

\subsection{Segmenters Based on Individual Feature Sets}

Segmenters Based on Individual Feature Sets.

\subsection{Segmenters Based on Feature Set combinations}

Segmenters Based on Feature Set combinations.

\section{Impact on Dialog Act Classification}

Impact on Dialog Act Classification.

\section{Discussion}

Discussion.


\chapter{Conclusion}

\label{ch:conclusions}

Dans ce chapitre, nous commençons par rappeler les apports de notre travail, avant de détailler comment nous pouvons l'améliorer dans le futur. Enfin, nous mentionnons les publications qui en ont été tirées.

\section{Réalisations}

La contribution principale de ce travail est une technique permettant d'exploiter les efforts cognitifs effectués par des humains attelés à une tâche de mise en forme de messages de réponse pour entraîner un segmenteur discursif.

Nous avons également développé un système de segmentation visant à soutenir l'analyse de messages en termes d'actes du langage et rapporté l'évaluation de différents modèles construits à partir d'ensembles de traits variés.

Enfin, nous avons proposé un nouveau corpus pour l'analyse de discussions asynchrones en ligne, qui a l'avantage d'être vaste, moderne, multimodal et multilingue.

\section{Perspectives}

Bien qu'il soit toujours possible de les améliorer, nos résultats indiquent que notre approche mérite un examen approfondi. Notre approche de segmentation reste relativement simple et peut facilement être étendue. Une manière de le faire serait de considérer les traits contextuels pour caractériser les phrases dans la structure originelle du message où elles ont été écrites.

Comme travaux futurs, nous prévoyons également de reproduire nos expériences sur un jeu de données constituées par l'ensemble des phrases des courriels, et pas seulement les phrases reprises dans les messages qui leur font réponse. Ce faisant, nous espérons corriger un biais dû fait que notre segmenteur n'est jusqu'à présent entraîné et testé que sur les parties des courriels typiquement reprises lors d'une conversation.

Enfin, il est possible de compléter nos expériences avec deux nouvelles approches pour l'évaluation. La première consistera à comparer la segmentation automatique avec celle effectuée par des annotateurs humains. Cette tâche reste difficile puisqu'il sera alors nécessaire de définir un protocole d'annotation, des lignes directrices et de construire de nouvelles ressources. La seconde évaluation que nous prévoyons d'effectuer est une évaluation extrinsèque. L'idée est de mesurer la contribution que peut apporter la segmentation d'un courriel au processus d'identification des actes du langage, c'est à dire de vérifier si la connaissance des frontières entre segments pourrait améliorer les systèmes de classification existants.

Pour aller dans une autre direction, nous pensons qu'il serait également possible d'exploiter notre approche pour estimer l'importance de tel ou tel segment, et peut-être lui trouver de nouvelles applications dans le cadre génération automatique de résumés par extraction de phrases.

\section{Publications}

Un article basé sur ce travail, titré \textit{Exploiting the human computational effort dedicated to message reply formatting for training discursive email segmenters}, a été soumis et accepté à \textit{The 8th Linguistic Annotation Workshop} (LAW 8), tenu en conjonction avec \textit{The 25th International Conference on Computational Linguistics} (COLING 2014). Cet article est joint à ce document en annexe.

\appendix
% appendices come here



\addcontentsline{toc}{chapter}{Bibliography}
\bibliographystyle{alpha}
\bibliography{bibliography}

\end{document}



\begin{document}

\title{\LARGE {\bf Exploiting the Human Computational Effort Dedicated to Message Reply Formatting for Training Discursive Email Segmenters}\\
 \vspace*{6mm}
}

\author{Soufian Salim}
\submitdate{July 2014}

\normallinespacing
\maketitle

\preface

\addcontentsline{toc}{chapter}{Abstract}

\begin{abstract}

In the context of multi-domain and multimodal online asynchronous discussion analysis, we propose an innovative strategy for manual annotation of dialog act (DA) segments. The process aims at supporting the analysis of messages in terms of DA. Our objective is to train a sequence labelling system to detect the segment boundaries.  The originality of the proposed approach is to avoid manually annotating the training data and instead exploit the human computational efforts dedicated to message reply formatting  when the writer replies to a message by inserting his response just after the quoted text appropriate to his intervention. We describe the approach, propose a new electronic mail corpus and report the evaluation of segmentation models we built.

\end{abstract}

\cleardoublepage

\addcontentsline{toc}{chapter}{Acknowledgements}

\begin{acknowledgements}

I would like to express gratitude to:

\begin{itemize}
 \item Nicolas Hernandez
 \vspace*{3mm}
 \item Christian Viard-Gaudin
 \vspace*{3mm}
 \item TALN team members at LINA
 \vspace*{3mm}
 \item ATAL teachers and students
\end{itemize}

\end{acknowledgements}

\body


\chapter{Introduction}

\label{ch:introduction}

\section{Contexte}

\lipsum[1]

\section{Motivation et objectifs}

\lipsum[1]

\section{Contributions}

\lipsum[1]


\chapter{Concepts et étude bibliographique}

\label{ch:background_and_related_work}

\section{Actes du langage}

\lipsum[1]

\section{Segmentation de textes}

\lipsum[1]

\section{Corpus de courriers électroniques}

\lipsum[1]


\chapter{Étiquetage automatique de corpus}

\label{ch:methodology_for_automatic_corpora_annotation}

Dans ce chapitre, nous présentons notre hypothèse ainsi que les étapes détaillées de notre approche pour l'étiquetage automatique de corpus.

\section{Hypothèses}

Nous partons du postulat que, lorsqu'un internaute reprend certains passages du courriel auquel il répond dans son message, il effectue des opérations cognitives pour identifier des fragments de texte autonomes et homogènes. Ces opérations peuvent être interpretées comme des opérations d'annotation. Les suppositions que l'on peut faire sur le type d'annotation dont il s'agit dépendent de l'opération qui a été effectuée. La suppression ou la reprise de texte original peut donner des indices sur la pertinence du contenu : du texte rejeté est probablement moins pertinent que du texte réutilisé.

\section{Schéma d'annotations}

Nous supposons que lorsqu'une personne ajoute du nouveau contenu entre deux blocs de texte cité, il effectue un découpage du message original. Nous supposons que la partie citée consiste en une unité d'information homogène. Par exemple, on peut supposer que la première phrase d'une partie citée comporte des instructions pour ouvrir un nouveau segment de discours tandis que la dernière phrase comporte des instructions pour achever le seegment. Par conséquent, nous pouvons effectuer certaines suppositions par rapport au role joué par ces phrases dans la structure informationnelle du message d'origine. On suppose qu'un message peut être divisé en segments du discours subsequents et consécutifs, chacun porteur de son propre acte du dialogue. On prend la phrase comme unité élémentaire. Une phrase dans un segment peut jouer l'un des roles suivants: \emph{starting and ending} (\textit{SE}), si elle constitue un segment à elle seule, \emph{starting} (\textit{S}), si elle débute un segment, \emph{inside} (\textit{I}), si elle n'est ni en début ni en fin de segment, et \emph{ending} (\textit{E}), si elle termine un segment.

Ce schéma est similaire au schéma \emph{BIO} à la différence qu'il est appliqué au niveau de la phrase et non au niveau du token \cite{ratinov:2009:conll}.

La figure~\ref{fig:exampleSegmentationLabels} illustre ce schéma en montrant comment les phrases de la figure~\ref{fig:exampleSourceReplyMessage} peuvent être alignées et comment les étiquettes peuvent en être inférées.

\section{Procédure de génération des données annotées}

Avant de pouvoir prédire les labels des phrases du message originel, il est nécessaire d'identifier celles qui ont été réutilisées dans un message de réponse. L'identification des lignes citées dans le message de réponse est insuffisant pour diverses raisons.

Premièrement, le segmenteur est supposé fonctionner sur des données non-bruitées (i.e. les nouveaux contenus dans les messages) alors qu'un texte cité est une version altérée du texte original. En effet, certains clients de messagerie électronique ne respectent pas toujours les standards et ne sont pas focément toujours compatibles\footnote{Les \textit{Request for Comments} (RFC) sont des règles et protocoles proposés par les groupes de travail participant à l'\textit{Internet Standardization} (\url{https://tools.ietf.org/html}). Certains RFC sont consacrés aux formats des courriels et aux spécifications d'encodage (voir RFC 2822 et 5335 pour commencer). Il y a eu de nombreuses propositions, parfois mises à jours et donc parfois rendues caduques, ce qui peut expliquer certains problèmes de compatibilité)}. En particulier, l'absence de certaines métadonnées peut causer le mauvais ré-encodage des blocs de citation à chaque échange. De plus, les programmes clients peuvent intégrer leurs propres mécanismes pour citer les précédents messages, ou encore tronquer les lignes trop longues\footnote{Fonctionnalité utilisée pour rendre le texte lisible sans avoir à scroller horizontalement. Les phrases sont généralement découpées en segments d'environ 80 caractères.}.

Deuxièmement, accéder aux messages originels peut permettre de prendre en compte certains traits contextuels (comme la disposition visuelle par exemple).

Troisièmement, pour aller plus loin, le context original du texte extrait contient également de l'information sur la segmentation d'un message. Par exemple, une phrase du message originel, qui ne serait pas présente dans la réponse, mais qui suit une phrase alignée, peut être considérée comme débutant un nouveau segment.

Don, en plus d'identifier les lignes citées, nous déployons une procédure d'alignement pour obtenir la version originale du texte cité. La procédure décrite étiquette les phrases de messages sources avec une information sur leur segmentation. Elle suit les étapes suivantes :

\begin{enumerate}
    \item Les messages postés dans le style interfolié sont identifiés
    \item Pour chaque paire message source / réponse :
    \begin{enumerate}
        \item Les deux messages sont tokenisés au niveau de la phrase et du mot (voir sous-section~\ref{subsec:tokenization} pour le détail des techniques employées pour la tokenisation)
        \item Les lignes de citations dans la réponse sont identifiées
        \item Les phrases qui font partie du texte cité dans le message de réponse sont identifiées
        \item Les phrases du message d'origine sont alignées avec le texte cité dans la réponse (voir sous-section~\ref{subsec:tokenization} pour le détail de la procédure d'alignement)
        \item Les phrases alignées sont étiquetées (voir sous-section~\ref{subsec:labelling} pour le détail de l'algorithme d'étiquetage)
        \item La séquence de phrases alignées est ajoutée au jeu de données
    \end{enumerate}
\end{enumerate}

Les messages contenant des messages à contenu interfolié sont reconnus grâce à la présence d'au moins deux lignes citées consécutives séparées par des lignes de nouveau contenu. Les paires de messages sources et leur réponse sont constituées à partir des champs \emph{in-reply-to} de leurs entêtes. Comme déclaré dans le RFC 3676\footnote{\url{http://www.ietf.org/rfc/rfc3676.txt}}, nous considérons comme des lignes citées les lignes commençant par le symbole "\textgreater" (chevron). Les lignes qui ne sont pas des lignes citées sont considérées comme étant des nouvelles lignes. Les tokens sont utilisés pour indexer les lignes citées et les phrases.

\subsection{Tokenization}

\label{subsec:tokenization}

\subsection{Alignement}

\label{subsec:alignment}

Pour trouver les alignements entre deux messages donnés, nous utilisons un algorithme d'alignement de chaînes basé sur la programmation dynamique (DP) \cite{sankoff:1983}. Dans le contexte de la reconnaissance de la parole, cet algorithme est aussi connu sous le nom de \textit{NIST align/scoring algorithm}. En effet il est largement utilisé pour évaluer les systèmes de reconnaissance de la parole en comparant leurs sorties au texte de référence. Il est utilisé en particulier pour calculer deux taux d'erreur : le \textit{Word Error Rate} (WER) et le \textit{Sentence Error rate} (SER).

L'algorithme fonctionne en effectuant une réduction de la distance de Levenshtein en attribuant aux mots corrects, aux insertions, aux suppressions et aux substitutions des poids respectifs de 0, 75, 75 et 100. L'algorithme est de complexité $O(MN)$.

L'Université de Carnegie Mellon fournit une implémentation de cet algorithme dans son kit de reconnaissance de la parole\footnote{Sphinx 4, $edu.cmu.sphinx.util.NISTAlign$, \url{http://cmusphinx.sourceforge.net}}

\subsection{Étiquetage}

\label{subsec:labelling}

L'étiquetage d'une phrase alignée (phrase du message source réutilisée dans la réponse) se fait suivant un simple algorithme à base de règles :

\begin{itemize}
    \item[\bullet] Pour chaque phrase source alignée :
    \begin{itemize}
        \item[\bullet] si la phrase est entourée par du nouveau contenu dans la réponse, l'étiquette est \texttt{Start\&End}
        \item[\bullet] sinon si la phrase est précédée par du nouveau contenu, l'étiquette est \texttt{Start}
        \item[\bullet] sinon si la phrase est suivie par du nouveau contenu, l'étiquette est \texttt{End}
        \item[\bullet] sinon, l'étiquette est \texttt{Inside}
    \end{itemize}
\end{itemize}

\begin{figure}
    \begin{minipage}{\textwidth}
        \fbox {
            \parbox{\linewidth}{
                \vspace{3mm}
                \small
                [Hi!]$^{S1}$\vspace{0.3cm}

                [I got my ubuntu cds today and i'm really impressed.]$^{S2}$ [My friends like them and my teachers too (i'm a student).]$^{S3}$ [It's really funny to see, how people like ubuntu and start feeling geek and blaming microsoft when they use it.]$^{S4}$ \vspace{0.3cm}

                [Unfortunately everyone wants an ubuntu cd, so can i download the cd covers anywhere or an 'official document' which i can attach to self-burned cds?]$^{S5}$\vspace{0.3cm}

                [I searched the entire web site but found nothing.]$^{S6}$ [Thanks in advance.]$^{S7}$\vspace{0.3cm}

                [John]$^{S8}$

                \vspace{3mm}
            }
        }

        \begin{center}
        Message source.
        \end{center}
        
        \fbox {
            \parbox{\linewidth}{
                \vspace{3mm}
                \small
                [On Sun, 04 Dec 2005, John Doe 
                \textless john@doe.com\textgreater wrote:]$^{R1}$\vspace{0.3cm}

                \textgreater [I got my ubuntu cds today and i'm really impressed.]$^{R2}$ [My friends like them and \\ \
                \textgreater my teachers too (i'm a student).]$^{R3}$ [It's really funny to see, how people like ubuntu \\ \
                \textgreater and start feeling geek and blaming microsoft when they use it.]$^{R4}$\vspace{0.3cm}

                [Rock!]$^{R5}$\vspace{0.3cm}

                \textgreater [Unfortunately everyone wants an ubuntu cd, so can i download the cd covers \\ \ 
                \textgreater anywhere or an 'official document' which i can attach to self-burned cds?]$^{R6}$\vspace{0.3cm}

                [We don't have any for the warty release, but we will have them for hoary, %\\ \ 
                because quite a few people have asked. :-)]$^{R7}$\vspace{0.3cm}

                [Bob.]$^{R8}$ %\vspace{0.1cm}
                \vspace{3mm}
            }
        }
        
        \begin{center}
        Message de réponse.
        \end{center}
    \end{minipage}

    \caption{Un message originel (ou "message source") et sa réponse (tirés de l'archive de courriers électroniques \textit{ubuntu-users}). Les différentes phrases ont été clairement indiquées.}
    \label{fig:exampleSourceReplyMessage}
\end{figure}

\begin{figure}
    \begin{minipage}{\textwidth}
        \small\centering
        \begin{tabular}{*{2}{|l}|c|}
        \hline
        \textbf{Source} & \textbf{Réponse} & \textbf{Étiquette}\\
            \hline
            S1  & & \\
            & R1 & \\
            \textit{S2}  & \textgreater \textit{R2}& \texttt{Start}\\
            \textit{S3}  & \textgreater \textit{R3}& \texttt{Inside}\\
            \textit{S4}  & \textgreater \textit{R4}& \texttt{End}\\
            & R5 & \\
            \textit{S5}  & \textgreater \textit{R6} & \texttt{Start\&End}\\
            & R7 & \\
            & [...] & \\
            S6 &  & \\ \ 
            [...] &  & \\
            \hline
        \end{tabular}
    \end{minipage}

    \caption{Alignement des phrases tirées des messages montrés dans la figure~\ref{fig:exampleSourceReplyMessage}, ainsi que les étiquettes inferrées de la reprise de texte du message source. Les étiquettes sont associées au phrases d'origine.}
    \label{fig:exampleSegmentationLabels}
\end{figure}


\chapter{Segmentation de courriels}

\label{ch:methodology_for_email_segmentation}

Dans ce chapitre, nous présentons notre approche pour la segmentation de courriels ainsi que les traits utilisés pour l’entraînement du classifieur.

\section{Étiquetage de séquences}

Nous choisissons de traiter le problème de la segmentation comme une tache d'étiquetage de séquences dont l'objectif est d'attribuer globalement le meilleur ensemble d'étiquettes pour la séquence entière d'un seul coup\footnote{Un exemple classique de tâche accomplie de cette manière est l'étiquetage morpho-syntaxique, qui cherche à identifier la nature grammaticale de chaque terme d'une phrase ou d'un document.}. Dans cette perspective, chaque courriel est traité comme une séquence de phrases. L'idée sous-jacente est que l'étiquette la plus pertinente pour une phrase est dépendante des traits et de l'étiquette des phrases proches. 

Notre segmenteur est basé sur un classifieur utilisant les champs aléatoires de Markov, tel qu'implémenté dans le programme d'étiquetage de séquences \textit{Wapiti} \cite{lavergne2010practical}. Nous fixons la taille de la fenêtre à 5, c'est à dire que l'algorithme prend en compte non seulement les traits de la phrase qu'il cherche à étiqueter mais également ceux des deux phrases précédentes et des deux phrases suivantes.

Entraîner le classifieur à reconnaître les différents labels du schéma d'annotation précédemment déterminé peut être problématique. En effet, il présente certains inconvénients qui peut nuire à l'efficacité du classifieur. En particulier, les phrases étiquetées \textit{SE} partageront, par définition, d'importantes caractéristiques avec les phrases étiquetées \textit{S} et \textit{E}. Nous choisissons donc de transformer ces annotations en un schéma binaire et nous contentons de différencier les phrases qui débutent un nouveau segment (\textit{True}), ou "phrases-frontières", de celles qui ne débutent pas un nouveau segment (\textit{False}). Le processus de conversion est trivial, et peut facilement être inversé.

Procédure de conversion :

\begin{itemize}
    \item[] Pour chaque phrase :
    \begin{itemize}
        \item[(a)] si la phrase est étiquetée \textit{SE} ou \textit{S}, l'étiquette devient \textit{True}
        \item[(b)] sinon, elle devient \textit{False}
    \end{itemize}
\end{itemize}

Procédure inverse, pour retrouver les étiquettes d'origine :

\begin{itemize}
    \item[] Pour chaque phrase :
    \begin{itemize}
        \item[(a)] si l'étiquette de la phrase courante est \textit{True} :
	    \begin{itemize}
	        \item[(i)] si la phrase suivante est étiquetée \textit{True}, elle devient \textit{SE}
	        \item[(ii)] sinon, elle devient \textit{S}
	    \end{itemize}
        \item[(b)] sinon :
	    \begin{itemize}
	        \item[(i)] si la phrase suivante est étiquetée \textit{True}, elle devient \textit{E}
	        \item[(ii)] sinon, elle devient \textit{I}
	    \end{itemize}
    \end{itemize}
\end{itemize}

\section{Ensembles de traits}

On distingue cinq ensembles de traits : les $n$-grammes, les traits basés sur la théorie de la structure de l'information, les traits thématiques, les traits stylistiques et les traits sémantiques (dans le cadre des expériences, les deux derniers ensembles sont regroupés sous l’appellation "traits divers"). Tous les traits sont indépendants du domaine et presque tous les traits sont indépendants du langage, à l'exception des traits sémantiques, qui peuvent néanmoins être facilement traduits.

Pour construire les traits du segmenteur, nous utilisons l'étiqueteur de Stanford pour l'étiquetage morpho-syntaxique \cite{toutanova2003feature}, et la base de données lexicale \textit{WordNet} pour la lemmatisation \cite{miller1995wordnet}.

\subsection{$n$-grammes}

On sélectionne, de manière insensible à la casse, les 1000\footnote{Valeur estimée empiriquement.} bigrammes et trigrammes apparaissant dans le plus grand nombre de phrases du corpus (ou \textit{document frequency}). Puisque la probabilité d'avoir de multiples occurrences d'un même $n$-gramme dans une phrase est extrêmement faible, nous ne conservons pas le nombre d'occurrences mais une valeur booléenne pour ne considérer que la présence ou l'absence du $n$-gramme.

\subsection{Traits basés sur la théorie de la structure de l'information}

Cet ensemble de traits est inspiré de la théorie de la structure de l'information, qui décrit l'information portée par une phrase en fonction de la façon dont elle est reliée à son contexte \cite{kruijff:1996}. La théorie affirme l'importance de constructions syntaxiques particulières et de l'ordre des mots dans la phrase. En effet pour des langages comme l'anglais ou le français, le début de la phrase est une position importante pour structurer l'information au niveau du discours, tandis que la fin de la phrase peut comporter de l'information utile pour annoncer ce qui vient ensuite. 

On s'intéresse aux trois premiers et derniers tokens significatifs de la phrase. Un token est considéré comme significatif si sa fréquence est supérieure à 1/2 000\footnote{Cette valeur a été déterminée empiriquement par rapport à nos données. Un travail supplémentaire devra être effectué pour la généraliser.}. Si une phrase contient moins de six tokens significatifs, le même token peut se retrouver dans les deux triplets. Si la phrase contient moins de trois tokens significatifs, les valeurs manquantes sont remplacées par une valeur spéciale "bouche-trou". Nous définissons trois traits individuels pour chacun des trois unigrammes, les deux bigrammes et le trigramme qui se trouvent dans chacun de ces triplets. Les traits sont les suivants : la forme de surface de chaque token (sensible à la casse), leur forme lemmatisée (insensible à la casse) et leur étiquette morpho-syntaxique. Ces traits sont illustrés par la figure \ref{fig:exampleSyntacticFeatures}.

\subsection{Trait thématique}

Le seul trait que nous prenons en compte pour la reconnaissance des variations thématiques est la sortie de l'algorithme \textit{TextTiling} \cite{hearst1997texttiling}. \textit{TextTiling} est l'un des algorithmes les plus communément utilisés pour la segmentation automatique de texte. Si l'algorithme détecte une rupture dans la cohésion lexicale du texte (entre deux blocs consécutifs), il place une frontière pour indiquer un changement thématique. En raison de la taille relativement courte des courriels, nous définissons la taille d'un bloc comme égale à trois fois la taille moyenne d'une phrase dans notre corpus. Nous définissons la "taille-étape" (la distance parcouru par la fenêtre glissante à chaque étape) comme égale à la taille moyenne d'une phrase du corpus.

\subsection{Traits divers}

Cet ensemble inclut les traits stylistiques et sémantiques. Il contient 24 traits, plusieurs ayant été empruntés à des travaux dans le domaine de la classification d'actes de dialogue \cite{qadir2011classifying} et de la segmentation de courriels \cite{lampert2009segmenting}. 

Les traits stylistiques capturent l'information portant sur la structure visuelle et la composition du message : 

\begin{itemize}
	\item[$\bullet$] la position de la phrase dans le courriel
	\item[$\bullet$] la taille moyenne des tokens
	\item[$\bullet$] le nombre total de tokens 
	\item[$\bullet$] le nombre total de caractères
	\item[$\bullet$] la proportion de majuscules
	\item[$\bullet$] la proportion de caractères alphabétiques
	\item[$\bullet$] la proportion de caractères numériques
	\item[$\bullet$] le nombre de chevrons
	\item[$\bullet$] si la phrase finit sur ou contient un point d'interrogation, une virgule ou un point-virgule
	\item[$\bullet$] si la phrase contient des caractères de ponctuation parmi ses trois premiers tokens (pour reconnaître les salutations \cite{qadir2011classifying}).
\end{itemize}

Les traits sémantiques cherchent à identifier certains mots et formules particuliers : 

\begin{itemize}
	\item[$\bullet$] si la phrase commence par un mot interrogatif de type "wh" (\textit{``who''}, \textit{``when''}, \textit{``where''}, \textit{``what''}, \textit{``which''}, \textit{``what''}, \textit{``how''})
	\item[$\bullet$] si la phrase contient un mot interrogatif de type "wh"
	\item[$\bullet$] si la phrase commence par une forme interrogative (e.g. "\textit{is it}", "\textit{are there}"...)
	\item[$\bullet$] si la phrase contient une forme interrogative
	\item[$\bullet$] si la phrase contient un modal (\textit{``can''}, \textit{``may''}, \textit{``must''}, \textit{``shall''}, \textit{``will''}, \textit{``might''}, \textit{``should''}, \textit{``would''}, \textit{``could''}, et leurs formes négatives)
	\item[$\bullet$] si la phrase contient une formule de planification (e.g. "\textit{I will}", "\textit{we are going to}"...)
	\item[$\bullet$] si la phrase contient des indices de la première personne (e.g. "\textit{we}", "\textit{my}"...)
	\item[$\bullet$] si la phrase contient des indices de la deuxième personne
	\item[$\bullet$] si la phrase contient des indices de la troisième personne
	\item[$\bullet$] le premier pronom personnel trouvé dans la phrase
	\item[$\bullet$] la première forme verbale rencontrée, telle qu'étiquetée par l'étiqueteur de Stanford, c'est à dire un élément du \textit{Penn Treebank tag set}\footnote{Liste alphabétique des étiquettes morpho-syntaxiques utilisées par le \textit{Penn Treebank Project} : \url{http://www.ling.upenn.edu/courses/Fall_2003/ling001/penn_treebank_pos.html}} (e.g. le trait \textit{``VBZ''} indique un verbe au présent et à la troisième personne du singulier).
\end{itemize}

\begin{table}\small\centering
	\begin{tabular}{*{2}{c}c}
		\toprule
		\textbf{Formes de surface} & \textbf{Lemmes} & \textbf{Étiquettes}\\
		\midrule
		Many & many & JJ\\
		thanks & thanks & NNS\\
		to & to & TO\\
		your & your & PRP\\
		suggestions & suggestion & DD\\
		. & . & .\\
		Many thanks & many thanks & JJ NNS\\
		thanks to . & thanks to . & NNS TO . \\
		your suggestions & your suggestion & PRP DD\\
		suggestions & suggestion & DD\\
		Many thanks to & many thanks to & JJ NNS TO\\
		your suggestions . & your suggestion . & PRP DD .\\
		\bottomrule
	\end{tabular}
	\caption{Traits syntaxiques formés par la phrase "\textit{Many thanks to all of you for the help you have offered, I have learned tremendously from all your suggestions}". Chaque cellule est un trait (36 au total).}
	\label{fig:exampleSyntacticFeatures}
\end{table}


\chapter{Experimental Framework}

\section{Corpora}

\subsection{Ubuntu Corpus}

Ubuntu Corpus.

\subsection{BC3 Corpus}

BC3 Corpus.

\section{Evaluation Protocol}

Evaluation Protocol.

\subsection{Methodology}

Methodology.

\subsection{Baselines}

Baselines.

\subsection{Metrics}

Metrics.


\chapter{Experiments and Results}

\section{10-fold cross validation}

\subsection{Preprocessing}

Preprocessing.

\subsection{Segmenters Based on Individual Feature Sets}

Segmenters Based on Individual Feature Sets.

\subsection{Segmenters Based on Feature Set combinations}

Segmenters Based on Feature Set combinations.

\section{Impact on Dialog Act Classification}

Impact on Dialog Act Classification.

\section{Discussion}

Discussion.


\chapter{Conclusion}

\label{ch:conclusions}

Dans ce chapitre, nous commençons par rappeler les apports de notre travail, avant de détailler comment nous pouvons l'améliorer dans le futur. Enfin, nous mentionnons les publications qui en ont été tirées.

\section{Réalisations}

La contribution principale de ce travail est une technique permettant d'exploiter les efforts cognitifs effectués par des humains attelés à une tâche de mise en forme de messages de réponse pour entraîner un segmenteur discursif.

Nous avons également développé un système de segmentation visant à soutenir l'analyse de messages en termes d'actes du langage et rapporté l'évaluation de différents modèles construits à partir d'ensembles de traits variés.

Enfin, nous avons proposé un nouveau corpus pour l'analyse de discussions asynchrones en ligne, qui a l'avantage d'être vaste, moderne, multimodal et multilingue.

\section{Perspectives}

Bien qu'il soit toujours possible de les améliorer, nos résultats indiquent que notre approche mérite un examen approfondi. Notre approche de segmentation reste relativement simple et peut facilement être étendue. Une manière de le faire serait de considérer les traits contextuels pour caractériser les phrases dans la structure originelle du message où elles ont été écrites.

Comme travaux futurs, nous prévoyons également de reproduire nos expériences sur un jeu de données constituées par l'ensemble des phrases des courriels, et pas seulement les phrases reprises dans les messages qui leur font réponse. Ce faisant, nous espérons corriger un biais dû fait que notre segmenteur n'est jusqu'à présent entraîné et testé que sur les parties des courriels typiquement reprises lors d'une conversation.

Enfin, il est possible de compléter nos expériences avec deux nouvelles approches pour l'évaluation. La première consistera à comparer la segmentation automatique avec celle effectuée par des annotateurs humains. Cette tâche reste difficile puisqu'il sera alors nécessaire de définir un protocole d'annotation, des lignes directrices et de construire de nouvelles ressources. La seconde évaluation que nous prévoyons d'effectuer est une évaluation extrinsèque. L'idée est de mesurer la contribution que peut apporter la segmentation d'un courriel au processus d'identification des actes du langage, c'est à dire de vérifier si la connaissance des frontières entre segments pourrait améliorer les systèmes de classification existants.

Pour aller dans une autre direction, nous pensons qu'il serait également possible d'exploiter notre approche pour estimer l'importance de tel ou tel segment, et peut-être lui trouver de nouvelles applications dans le cadre génération automatique de résumés par extraction de phrases.

\section{Publications}

Un article basé sur ce travail, titré \textit{Exploiting the human computational effort dedicated to message reply formatting for training discursive email segmenters}, a été soumis et accepté à \textit{The 8th Linguistic Annotation Workshop} (LAW 8), tenu en conjonction avec \textit{The 25th International Conference on Computational Linguistics} (COLING 2014). Cet article est joint à ce document en annexe.

\appendix
% appendices come here



\addcontentsline{toc}{chapter}{Bibliography}
\bibliographystyle{alpha}
\bibliography{bibliography}

\end{document}


\begin{document}

\title{\LARGE {\bf Segmentation discursive des courriels selon une approche supervisée "paresseuse"}\\
	\vspace*{6mm}
}

\author{Soufian Salim}
\submitdate{Juillet 2014}

\normallinespacing
\maketitle

\preface

\begin{resume}

\addcontentsline{toc}{chapter}{Résumé}

Dans le cadre de l'analyse de discussions asynchrones en ligne, multi-modales et multi-domaines, nous proposons une stratégie novatrice pour l'identification de segments d'actes de langage. Le processus décrit vise à soutenir l'analyse de messages en termes d'intention communicative. Notre objectif est de développer un système d'étiquetage de séquences permettant de détecter les frontières entre segments. L'originalité de l'approche proposée vient du fait que nous exploitons les efforts cognitifs effectués par des humains pour la tache de mise en forme de messages de réponse pour éviter d'avoir à effectuer un laborieux travail d'annotation manuelle. Nous décrivons notre approche, proposons un nouveau corpus de courriers électroniques et rapportons l'évaluation des modèles de segmentation ainsi construits.

\end{resume}

\begin{abstract}

\addcontentsline{toc}{chapter}{Abstract}

In the context of multi-domain and multimodal online asynchronous discussion analysis, we propose an innovative strategy for the annotation of speech act (SA) segments. The process aims at supporting the analysis of messages in terms of SA. Our objective is to train a sequence labelling system to detect the segment boundaries.  The originality of the proposed approach is to avoid manually annotating the training data and instead exploit the human computational efforts dedicated to message reply formatting  when the writer replies to a message by inserting his response just after the quoted text appropriate to his intervention. We describe the approach, propose a new electronic mail corpus and report the evaluation of segmentation models we built.

\end{abstract} 

\begin{acknowledgements}

\addcontentsline{toc}{chapter}{Remerciements}

Je voudrais remercier :

\begin{itemize}
	\item Nicolas Hernandez (LINA)
	\vspace*{3mm}
	\item Christian Viard-Gaudin (IRCCyN)
	\vspace*{3mm}
	\item Nathalie Camelin (LIUM)
	\vspace*{3mm}
	\item Les membres de l'équipe TALN au LINA
	\vspace*{3mm}
	\item Les enseignants et étudiants du master ATAL
\end{itemize}

\end{acknowledgements}

\body


\chapter{Introduction}

\label{ch:introduction}

\section{Contexte}

\lipsum[1]

\section{Motivation et objectifs}

\lipsum[1]

\section{Contributions}

\lipsum[1]


\chapter{Concepts et étude bibliographique}

\label{ch:background_and_related_work}

\section{Actes du langage}

\lipsum[1]

\section{Segmentation de textes}

\lipsum[1]

\section{Corpus de courriers électroniques}

\lipsum[1]


\chapter{Étiquetage automatique de corpus}

\label{ch:methodology_for_automatic_corpora_annotation}

Dans ce chapitre, nous présentons notre hypothèse ainsi que les étapes détaillées de notre approche pour l'étiquetage automatique de corpus.

\section{Hypothèses}

Nous partons du postulat que, lorsqu'un internaute reprend certains passages du courriel auquel il répond dans son message, il effectue des opérations cognitives pour identifier des fragments de texte autonomes et homogènes. Ces opérations peuvent être interpretées comme des opérations d'annotation. Les suppositions que l'on peut faire sur le type d'annotation dont il s'agit dépendent de l'opération qui a été effectuée. La suppression ou la reprise de texte original peut donner des indices sur la pertinence du contenu : du texte rejeté est probablement moins pertinent que du texte réutilisé.

\section{Schéma d'annotations}

Nous supposons que lorsqu'une personne ajoute du nouveau contenu entre deux blocs de texte cité, il effectue un découpage du message original. Nous supposons que la partie citée consiste en une unité d'information homogène. Par exemple, on peut supposer que la première phrase d'une partie citée comporte des instructions pour ouvrir un nouveau segment de discours tandis que la dernière phrase comporte des instructions pour achever le seegment. Par conséquent, nous pouvons effectuer certaines suppositions par rapport au role joué par ces phrases dans la structure informationnelle du message d'origine. On suppose qu'un message peut être divisé en segments du discours subsequents et consécutifs, chacun porteur de son propre acte du dialogue. On prend la phrase comme unité élémentaire. Une phrase dans un segment peut jouer l'un des roles suivants: \emph{starting and ending} (\textit{SE}), si elle constitue un segment à elle seule, \emph{starting} (\textit{S}), si elle débute un segment, \emph{inside} (\textit{I}), si elle n'est ni en début ni en fin de segment, et \emph{ending} (\textit{E}), si elle termine un segment.

Ce schéma est similaire au schéma \emph{BIO} à la différence qu'il est appliqué au niveau de la phrase et non au niveau du token \cite{ratinov:2009:conll}.

La figure~\ref{fig:exampleSegmentationLabels} illustre ce schéma en montrant comment les phrases de la figure~\ref{fig:exampleSourceReplyMessage} peuvent être alignées et comment les étiquettes peuvent en être inférées.

\section{Procédure de génération des données annotées}

Avant de pouvoir prédire les labels des phrases du message originel, il est nécessaire d'identifier celles qui ont été réutilisées dans un message de réponse. L'identification des lignes citées dans le message de réponse est insuffisant pour diverses raisons.

Premièrement, le segmenteur est supposé fonctionner sur des données non-bruitées (i.e. les nouveaux contenus dans les messages) alors qu'un texte cité est une version altérée du texte original. En effet, certains clients de messagerie électronique ne respectent pas toujours les standards et ne sont pas focément toujours compatibles\footnote{Les \textit{Request for Comments} (RFC) sont des règles et protocoles proposés par les groupes de travail participant à l'\textit{Internet Standardization} (\url{https://tools.ietf.org/html}). Certains RFC sont consacrés aux formats des courriels et aux spécifications d'encodage (voir RFC 2822 et 5335 pour commencer). Il y a eu de nombreuses propositions, parfois mises à jours et donc parfois rendues caduques, ce qui peut expliquer certains problèmes de compatibilité)}. En particulier, l'absence de certaines métadonnées peut causer le mauvais ré-encodage des blocs de citation à chaque échange. De plus, les programmes clients peuvent intégrer leurs propres mécanismes pour citer les précédents messages, ou encore tronquer les lignes trop longues\footnote{Fonctionnalité utilisée pour rendre le texte lisible sans avoir à scroller horizontalement. Les phrases sont généralement découpées en segments d'environ 80 caractères.}.

Deuxièmement, accéder aux messages originels peut permettre de prendre en compte certains traits contextuels (comme la disposition visuelle par exemple).

Troisièmement, pour aller plus loin, le context original du texte extrait contient également de l'information sur la segmentation d'un message. Par exemple, une phrase du message originel, qui ne serait pas présente dans la réponse, mais qui suit une phrase alignée, peut être considérée comme débutant un nouveau segment.

Don, en plus d'identifier les lignes citées, nous déployons une procédure d'alignement pour obtenir la version originale du texte cité. La procédure décrite étiquette les phrases de messages sources avec une information sur leur segmentation. Elle suit les étapes suivantes :

\begin{enumerate}
    \item Les messages postés dans le style interfolié sont identifiés
    \item Pour chaque paire message source / réponse :
    \begin{enumerate}
        \item Les deux messages sont tokenisés au niveau de la phrase et du mot (voir sous-section~\ref{subsec:tokenization} pour le détail des techniques employées pour la tokenisation)
        \item Les lignes de citations dans la réponse sont identifiées
        \item Les phrases qui font partie du texte cité dans le message de réponse sont identifiées
        \item Les phrases du message d'origine sont alignées avec le texte cité dans la réponse (voir sous-section~\ref{subsec:tokenization} pour le détail de la procédure d'alignement)
        \item Les phrases alignées sont étiquetées (voir sous-section~\ref{subsec:labelling} pour le détail de l'algorithme d'étiquetage)
        \item La séquence de phrases alignées est ajoutée au jeu de données
    \end{enumerate}
\end{enumerate}

Les messages contenant des messages à contenu interfolié sont reconnus grâce à la présence d'au moins deux lignes citées consécutives séparées par des lignes de nouveau contenu. Les paires de messages sources et leur réponse sont constituées à partir des champs \emph{in-reply-to} de leurs entêtes. Comme déclaré dans le RFC 3676\footnote{\url{http://www.ietf.org/rfc/rfc3676.txt}}, nous considérons comme des lignes citées les lignes commençant par le symbole "\textgreater" (chevron). Les lignes qui ne sont pas des lignes citées sont considérées comme étant des nouvelles lignes. Les tokens sont utilisés pour indexer les lignes citées et les phrases.

\subsection{Tokenization}

\label{subsec:tokenization}

\subsection{Alignement}

\label{subsec:alignment}

Pour trouver les alignements entre deux messages donnés, nous utilisons un algorithme d'alignement de chaînes basé sur la programmation dynamique (DP) \cite{sankoff:1983}. Dans le contexte de la reconnaissance de la parole, cet algorithme est aussi connu sous le nom de \textit{NIST align/scoring algorithm}. En effet il est largement utilisé pour évaluer les systèmes de reconnaissance de la parole en comparant leurs sorties au texte de référence. Il est utilisé en particulier pour calculer deux taux d'erreur : le \textit{Word Error Rate} (WER) et le \textit{Sentence Error rate} (SER).

L'algorithme fonctionne en effectuant une réduction de la distance de Levenshtein en attribuant aux mots corrects, aux insertions, aux suppressions et aux substitutions des poids respectifs de 0, 75, 75 et 100. L'algorithme est de complexité $O(MN)$.

L'Université de Carnegie Mellon fournit une implémentation de cet algorithme dans son kit de reconnaissance de la parole\footnote{Sphinx 4, $edu.cmu.sphinx.util.NISTAlign$, \url{http://cmusphinx.sourceforge.net}}

\subsection{Étiquetage}

\label{subsec:labelling}

L'étiquetage d'une phrase alignée (phrase du message source réutilisée dans la réponse) se fait suivant un simple algorithme à base de règles :

\begin{itemize}
    \item[\bullet] Pour chaque phrase source alignée :
    \begin{itemize}
        \item[\bullet] si la phrase est entourée par du nouveau contenu dans la réponse, l'étiquette est \texttt{Start\&End}
        \item[\bullet] sinon si la phrase est précédée par du nouveau contenu, l'étiquette est \texttt{Start}
        \item[\bullet] sinon si la phrase est suivie par du nouveau contenu, l'étiquette est \texttt{End}
        \item[\bullet] sinon, l'étiquette est \texttt{Inside}
    \end{itemize}
\end{itemize}

\begin{figure}
    \begin{minipage}{\textwidth}
        \fbox {
            \parbox{\linewidth}{
                \vspace{3mm}
                \small
                [Hi!]$^{S1}$\vspace{0.3cm}

                [I got my ubuntu cds today and i'm really impressed.]$^{S2}$ [My friends like them and my teachers too (i'm a student).]$^{S3}$ [It's really funny to see, how people like ubuntu and start feeling geek and blaming microsoft when they use it.]$^{S4}$ \vspace{0.3cm}

                [Unfortunately everyone wants an ubuntu cd, so can i download the cd covers anywhere or an 'official document' which i can attach to self-burned cds?]$^{S5}$\vspace{0.3cm}

                [I searched the entire web site but found nothing.]$^{S6}$ [Thanks in advance.]$^{S7}$\vspace{0.3cm}

                [John]$^{S8}$

                \vspace{3mm}
            }
        }

        \begin{center}
        Message source.
        \end{center}
        
        \fbox {
            \parbox{\linewidth}{
                \vspace{3mm}
                \small
                [On Sun, 04 Dec 2005, John Doe 
                \textless john@doe.com\textgreater wrote:]$^{R1}$\vspace{0.3cm}

                \textgreater [I got my ubuntu cds today and i'm really impressed.]$^{R2}$ [My friends like them and \\ \
                \textgreater my teachers too (i'm a student).]$^{R3}$ [It's really funny to see, how people like ubuntu \\ \
                \textgreater and start feeling geek and blaming microsoft when they use it.]$^{R4}$\vspace{0.3cm}

                [Rock!]$^{R5}$\vspace{0.3cm}

                \textgreater [Unfortunately everyone wants an ubuntu cd, so can i download the cd covers \\ \ 
                \textgreater anywhere or an 'official document' which i can attach to self-burned cds?]$^{R6}$\vspace{0.3cm}

                [We don't have any for the warty release, but we will have them for hoary, %\\ \ 
                because quite a few people have asked. :-)]$^{R7}$\vspace{0.3cm}

                [Bob.]$^{R8}$ %\vspace{0.1cm}
                \vspace{3mm}
            }
        }
        
        \begin{center}
        Message de réponse.
        \end{center}
    \end{minipage}

    \caption{Un message originel (ou "message source") et sa réponse (tirés de l'archive de courriers électroniques \textit{ubuntu-users}). Les différentes phrases ont été clairement indiquées.}
    \label{fig:exampleSourceReplyMessage}
\end{figure}

\begin{figure}
    \begin{minipage}{\textwidth}
        \small\centering
        \begin{tabular}{*{2}{|l}|c|}
        \hline
        \textbf{Source} & \textbf{Réponse} & \textbf{Étiquette}\\
            \hline
            S1  & & \\
            & R1 & \\
            \textit{S2}  & \textgreater \textit{R2}& \texttt{Start}\\
            \textit{S3}  & \textgreater \textit{R3}& \texttt{Inside}\\
            \textit{S4}  & \textgreater \textit{R4}& \texttt{End}\\
            & R5 & \\
            \textit{S5}  & \textgreater \textit{R6} & \texttt{Start\&End}\\
            & R7 & \\
            & [...] & \\
            S6 &  & \\ \ 
            [...] &  & \\
            \hline
        \end{tabular}
    \end{minipage}

    \caption{Alignement des phrases tirées des messages montrés dans la figure~\ref{fig:exampleSourceReplyMessage}, ainsi que les étiquettes inferrées de la reprise de texte du message source. Les étiquettes sont associées au phrases d'origine.}
    \label{fig:exampleSegmentationLabels}
\end{figure}


\chapter{Segmentation de courriels}

\label{ch:methodology_for_email_segmentation}

Dans ce chapitre, nous présentons notre approche pour la segmentation de courriels ainsi que les traits utilisés pour l’entraînement du classifieur.

\section{Étiquetage de séquences}

Nous choisissons de traiter le problème de la segmentation comme une tache d'étiquetage de séquences dont l'objectif est d'attribuer globalement le meilleur ensemble d'étiquettes pour la séquence entière d'un seul coup\footnote{Un exemple classique de tâche accomplie de cette manière est l'étiquetage morpho-syntaxique, qui cherche à identifier la nature grammaticale de chaque terme d'une phrase ou d'un document.}. Dans cette perspective, chaque courriel est traité comme une séquence de phrases. L'idée sous-jacente est que l'étiquette la plus pertinente pour une phrase est dépendante des traits et de l'étiquette des phrases proches. 

Notre segmenteur est basé sur un classifieur utilisant les champs aléatoires de Markov, tel qu'implémenté dans le programme d'étiquetage de séquences \textit{Wapiti} \cite{lavergne2010practical}. Nous fixons la taille de la fenêtre à 5, c'est à dire que l'algorithme prend en compte non seulement les traits de la phrase qu'il cherche à étiqueter mais également ceux des deux phrases précédentes et des deux phrases suivantes.

Entraîner le classifieur à reconnaître les différents labels du schéma d'annotation précédemment déterminé peut être problématique. En effet, il présente certains inconvénients qui peut nuire à l'efficacité du classifieur. En particulier, les phrases étiquetées \textit{SE} partageront, par définition, d'importantes caractéristiques avec les phrases étiquetées \textit{S} et \textit{E}. Nous choisissons donc de transformer ces annotations en un schéma binaire et nous contentons de différencier les phrases qui débutent un nouveau segment (\textit{True}), ou "phrases-frontières", de celles qui ne débutent pas un nouveau segment (\textit{False}). Le processus de conversion est trivial, et peut facilement être inversé.

Procédure de conversion :

\begin{itemize}
    \item[] Pour chaque phrase :
    \begin{itemize}
        \item[(a)] si la phrase est étiquetée \textit{SE} ou \textit{S}, l'étiquette devient \textit{True}
        \item[(b)] sinon, elle devient \textit{False}
    \end{itemize}
\end{itemize}

Procédure inverse, pour retrouver les étiquettes d'origine :

\begin{itemize}
    \item[] Pour chaque phrase :
    \begin{itemize}
        \item[(a)] si l'étiquette de la phrase courante est \textit{True} :
	    \begin{itemize}
	        \item[(i)] si la phrase suivante est étiquetée \textit{True}, elle devient \textit{SE}
	        \item[(ii)] sinon, elle devient \textit{S}
	    \end{itemize}
        \item[(b)] sinon :
	    \begin{itemize}
	        \item[(i)] si la phrase suivante est étiquetée \textit{True}, elle devient \textit{E}
	        \item[(ii)] sinon, elle devient \textit{I}
	    \end{itemize}
    \end{itemize}
\end{itemize}

\section{Ensembles de traits}

On distingue cinq ensembles de traits : les $n$-grammes, les traits basés sur la théorie de la structure de l'information, les traits thématiques, les traits stylistiques et les traits sémantiques (dans le cadre des expériences, les deux derniers ensembles sont regroupés sous l’appellation "traits divers"). Tous les traits sont indépendants du domaine et presque tous les traits sont indépendants du langage, à l'exception des traits sémantiques, qui peuvent néanmoins être facilement traduits.

Pour construire les traits du segmenteur, nous utilisons l'étiqueteur de Stanford pour l'étiquetage morpho-syntaxique \cite{toutanova2003feature}, et la base de données lexicale \textit{WordNet} pour la lemmatisation \cite{miller1995wordnet}.

\subsection{$n$-grammes}

On sélectionne, de manière insensible à la casse, les 1000\footnote{Valeur estimée empiriquement.} bigrammes et trigrammes apparaissant dans le plus grand nombre de phrases du corpus (ou \textit{document frequency}). Puisque la probabilité d'avoir de multiples occurrences d'un même $n$-gramme dans une phrase est extrêmement faible, nous ne conservons pas le nombre d'occurrences mais une valeur booléenne pour ne considérer que la présence ou l'absence du $n$-gramme.

\subsection{Traits basés sur la théorie de la structure de l'information}

Cet ensemble de traits est inspiré de la théorie de la structure de l'information, qui décrit l'information portée par une phrase en fonction de la façon dont elle est reliée à son contexte \cite{kruijff:1996}. La théorie affirme l'importance de constructions syntaxiques particulières et de l'ordre des mots dans la phrase. En effet pour des langages comme l'anglais ou le français, le début de la phrase est une position importante pour structurer l'information au niveau du discours, tandis que la fin de la phrase peut comporter de l'information utile pour annoncer ce qui vient ensuite. 

On s'intéresse aux trois premiers et derniers tokens significatifs de la phrase. Un token est considéré comme significatif si sa fréquence est supérieure à 1/2 000\footnote{Cette valeur a été déterminée empiriquement par rapport à nos données. Un travail supplémentaire devra être effectué pour la généraliser.}. Si une phrase contient moins de six tokens significatifs, le même token peut se retrouver dans les deux triplets. Si la phrase contient moins de trois tokens significatifs, les valeurs manquantes sont remplacées par une valeur spéciale "bouche-trou". Nous définissons trois traits individuels pour chacun des trois unigrammes, les deux bigrammes et le trigramme qui se trouvent dans chacun de ces triplets. Les traits sont les suivants : la forme de surface de chaque token (sensible à la casse), leur forme lemmatisée (insensible à la casse) et leur étiquette morpho-syntaxique. Ces traits sont illustrés par la figure \ref{fig:exampleSyntacticFeatures}.

\subsection{Trait thématique}

Le seul trait que nous prenons en compte pour la reconnaissance des variations thématiques est la sortie de l'algorithme \textit{TextTiling} \cite{hearst1997texttiling}. \textit{TextTiling} est l'un des algorithmes les plus communément utilisés pour la segmentation automatique de texte. Si l'algorithme détecte une rupture dans la cohésion lexicale du texte (entre deux blocs consécutifs), il place une frontière pour indiquer un changement thématique. En raison de la taille relativement courte des courriels, nous définissons la taille d'un bloc comme égale à trois fois la taille moyenne d'une phrase dans notre corpus. Nous définissons la "taille-étape" (la distance parcouru par la fenêtre glissante à chaque étape) comme égale à la taille moyenne d'une phrase du corpus.

\subsection{Traits divers}

Cet ensemble inclut les traits stylistiques et sémantiques. Il contient 24 traits, plusieurs ayant été empruntés à des travaux dans le domaine de la classification d'actes de dialogue \cite{qadir2011classifying} et de la segmentation de courriels \cite{lampert2009segmenting}. 

Les traits stylistiques capturent l'information portant sur la structure visuelle et la composition du message : 

\begin{itemize}
	\item[$\bullet$] la position de la phrase dans le courriel
	\item[$\bullet$] la taille moyenne des tokens
	\item[$\bullet$] le nombre total de tokens 
	\item[$\bullet$] le nombre total de caractères
	\item[$\bullet$] la proportion de majuscules
	\item[$\bullet$] la proportion de caractères alphabétiques
	\item[$\bullet$] la proportion de caractères numériques
	\item[$\bullet$] le nombre de chevrons
	\item[$\bullet$] si la phrase finit sur ou contient un point d'interrogation, une virgule ou un point-virgule
	\item[$\bullet$] si la phrase contient des caractères de ponctuation parmi ses trois premiers tokens (pour reconnaître les salutations \cite{qadir2011classifying}).
\end{itemize}

Les traits sémantiques cherchent à identifier certains mots et formules particuliers : 

\begin{itemize}
	\item[$\bullet$] si la phrase commence par un mot interrogatif de type "wh" (\textit{``who''}, \textit{``when''}, \textit{``where''}, \textit{``what''}, \textit{``which''}, \textit{``what''}, \textit{``how''})
	\item[$\bullet$] si la phrase contient un mot interrogatif de type "wh"
	\item[$\bullet$] si la phrase commence par une forme interrogative (e.g. "\textit{is it}", "\textit{are there}"...)
	\item[$\bullet$] si la phrase contient une forme interrogative
	\item[$\bullet$] si la phrase contient un modal (\textit{``can''}, \textit{``may''}, \textit{``must''}, \textit{``shall''}, \textit{``will''}, \textit{``might''}, \textit{``should''}, \textit{``would''}, \textit{``could''}, et leurs formes négatives)
	\item[$\bullet$] si la phrase contient une formule de planification (e.g. "\textit{I will}", "\textit{we are going to}"...)
	\item[$\bullet$] si la phrase contient des indices de la première personne (e.g. "\textit{we}", "\textit{my}"...)
	\item[$\bullet$] si la phrase contient des indices de la deuxième personne
	\item[$\bullet$] si la phrase contient des indices de la troisième personne
	\item[$\bullet$] le premier pronom personnel trouvé dans la phrase
	\item[$\bullet$] la première forme verbale rencontrée, telle qu'étiquetée par l'étiqueteur de Stanford, c'est à dire un élément du \textit{Penn Treebank tag set}\footnote{Liste alphabétique des étiquettes morpho-syntaxiques utilisées par le \textit{Penn Treebank Project} : \url{http://www.ling.upenn.edu/courses/Fall_2003/ling001/penn_treebank_pos.html}} (e.g. le trait \textit{``VBZ''} indique un verbe au présent et à la troisième personne du singulier).
\end{itemize}

\begin{table}\small\centering
	\begin{tabular}{*{2}{c}c}
		\toprule
		\textbf{Formes de surface} & \textbf{Lemmes} & \textbf{Étiquettes}\\
		\midrule
		Many & many & JJ\\
		thanks & thanks & NNS\\
		to & to & TO\\
		your & your & PRP\\
		suggestions & suggestion & DD\\
		. & . & .\\
		Many thanks & many thanks & JJ NNS\\
		thanks to . & thanks to . & NNS TO . \\
		your suggestions & your suggestion & PRP DD\\
		suggestions & suggestion & DD\\
		Many thanks to & many thanks to & JJ NNS TO\\
		your suggestions . & your suggestion . & PRP DD .\\
		\bottomrule
	\end{tabular}
	\caption{Traits syntaxiques formés par la phrase "\textit{Many thanks to all of you for the help you have offered, I have learned tremendously from all your suggestions}". Chaque cellule est un trait (36 au total).}
	\label{fig:exampleSyntacticFeatures}
\end{table}


\chapter{Experimental Framework}

\section{Corpora}

\subsection{Ubuntu Corpus}

Ubuntu Corpus.

\subsection{BC3 Corpus}

BC3 Corpus.

\section{Evaluation Protocol}

Evaluation Protocol.

\subsection{Methodology}

Methodology.

\subsection{Baselines}

Baselines.

\subsection{Metrics}

Metrics.


\chapter{Experiments and Results}

\section{10-fold cross validation}

\subsection{Preprocessing}

Preprocessing.

\subsection{Segmenters Based on Individual Feature Sets}

Segmenters Based on Individual Feature Sets.

\subsection{Segmenters Based on Feature Set combinations}

Segmenters Based on Feature Set combinations.

\section{Impact on Dialog Act Classification}

Impact on Dialog Act Classification.

\section{Discussion}

Discussion.


\chapter{Conclusion}

\label{ch:conclusions}

Dans ce chapitre, nous commençons par rappeler les apports de notre travail, avant de détailler comment nous pouvons l'améliorer dans le futur. Enfin, nous mentionnons les publications qui en ont été tirées.

\section{Réalisations}

La contribution principale de ce travail est une technique permettant d'exploiter les efforts cognitifs effectués par des humains attelés à une tâche de mise en forme de messages de réponse pour entraîner un segmenteur discursif.

Nous avons également développé un système de segmentation visant à soutenir l'analyse de messages en termes d'actes du langage et rapporté l'évaluation de différents modèles construits à partir d'ensembles de traits variés.

Enfin, nous avons proposé un nouveau corpus pour l'analyse de discussions asynchrones en ligne, qui a l'avantage d'être vaste, moderne, multimodal et multilingue.

\section{Perspectives}

Bien qu'il soit toujours possible de les améliorer, nos résultats indiquent que notre approche mérite un examen approfondi. Notre approche de segmentation reste relativement simple et peut facilement être étendue. Une manière de le faire serait de considérer les traits contextuels pour caractériser les phrases dans la structure originelle du message où elles ont été écrites.

Comme travaux futurs, nous prévoyons également de reproduire nos expériences sur un jeu de données constituées par l'ensemble des phrases des courriels, et pas seulement les phrases reprises dans les messages qui leur font réponse. Ce faisant, nous espérons corriger un biais dû fait que notre segmenteur n'est jusqu'à présent entraîné et testé que sur les parties des courriels typiquement reprises lors d'une conversation.

Enfin, il est possible de compléter nos expériences avec deux nouvelles approches pour l'évaluation. La première consistera à comparer la segmentation automatique avec celle effectuée par des annotateurs humains. Cette tâche reste difficile puisqu'il sera alors nécessaire de définir un protocole d'annotation, des lignes directrices et de construire de nouvelles ressources. La seconde évaluation que nous prévoyons d'effectuer est une évaluation extrinsèque. L'idée est de mesurer la contribution que peut apporter la segmentation d'un courriel au processus d'identification des actes du langage, c'est à dire de vérifier si la connaissance des frontières entre segments pourrait améliorer les systèmes de classification existants.

Pour aller dans une autre direction, nous pensons qu'il serait également possible d'exploiter notre approche pour estimer l'importance de tel ou tel segment, et peut-être lui trouver de nouvelles applications dans le cadre génération automatique de résumés par extraction de phrases.

\section{Publications}

Un article basé sur ce travail, titré \textit{Exploiting the human computational effort dedicated to message reply formatting for training discursive email segmenters}, a été soumis et accepté à \textit{The 8th Linguistic Annotation Workshop} (LAW 8), tenu en conjonction avec \textit{The 25th International Conference on Computational Linguistics} (COLING 2014). Cet article est joint à ce document en annexe.

\begin{appendices}

\chapter{\textit{Exploiting the human computational effort dedicated to message reply formatting for training discursive email segmenters}}

Article accepté à \textit{The 8th Linguistic Annotation Workshop} (LAW 8), tenu en conjonction avec \textit{The 25th International Conference on Computational Linguistics} (COLING 2014).

\includepdf[pages=-]{law8.pdf}

\end{appendices}

\addcontentsline{toc}{chapter}{Bibliographie}
\bibliographystyle{authordate2}
\bibliography{bibliography}

\end{document}
